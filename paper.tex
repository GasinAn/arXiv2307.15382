\documentclass{article}
\usepackage{amsmath}
\def\p{\partial}
\begin{document}
If we only take the lowest order effects of the motion of mass sources into account and neglect stresses, we can define two vector field, $\vec{E}$ and $\vec{B}$, which are similar to the electric and magnetic field. We can define $A_\mu=-\frac{1}{4}c_0\bar{h}_{0 \mu}=(\varphi/c_0,\vec{A})$, then $A_\mu$, which are similar to the electromagnetic 4-potential, will satisfy
\begin{equation}
    \p^\nu\p_\nu A_\mu=-\frac{4\pi G_0}{c_0^2}J_\mu,\quad\p^\mu A_\mu=0,
\end{equation}
where $J_\mu=-c_0T_{0 \mu}/c_0^2=(c_0\rho,\vec{j})=\rho(c_0,\vec{v})$ is 4-momentum density. If we define $\vec{E}=-\vec{\nabla}\varphi-\frac{\p}{\p t}\vec{A}$, $\vec{B}=\vec{\nabla}\times\vec{A}$, and $\varepsilon_{\text{G}0}=\frac{1}{4\pi G_0}$, $\mu_{\text{G}0}=\frac{4\pi G_0}{c_0^2}$, then there will be
\begin{equation}\label{maxwell_0}
    \begin{cases}
        \vec{\nabla}\cdot(\varepsilon_{\text{G}0}\vec{E})=\rho,\\
        \vec{\nabla}\cdot\vec{B}=0,\\
        \vec{\nabla}\times\vec{E}=-\frac{\p}{\p t}\vec{B},\\
        \vec{\nabla}\times(\mu_{\text{G}0}^{-1}\vec{B})=\vec{j}+\frac{\p}{\p t}(\varepsilon_{\text{G}0}\vec{E}),
    \end{cases}
\end{equation}
and the acceleration of mass particle $\vec{a}$ will satisfy
\begin{equation}\label{lorentz}
    \vec{a}=-\vec{E}-4\vec{v}\times\vec{B}.
\end{equation}

Since \eqref{maxwell_0} are identical to Maxwell's equations, and \eqref{lorentz} is identical to the Lorentz force equation except for an overall minus sign and a factor of $4$ in the ``magnetic force'' term, we assume that the constants $\varepsilon_{\text{G}0}$ and $\mu_{\text{G}0}$ can simply vary as constants $\varepsilon_{0}$ and $\mu_{0}$ in electromagnetic theories, which means, \eqref{lorentz} is still valid and \eqref{maxwell_0} becomes
\begin{equation}\label{maxwell}
    \begin{cases}
        \vec{\nabla}\cdot\vec{D}=\rho,\\
        \vec{\nabla}\cdot\vec{B}=0,\\
        \vec{\nabla}\times\vec{E}=-\frac{\p}{\p t}\vec{B},\\
        \vec{\nabla}\times\vec{H}=\vec{j}+\frac{\p}{\p t}\vec{D},
    \end{cases}
\end{equation}
where $\vec{D}=\varepsilon_{\text{G}}\vec{E}$, $\vec{B}=\mu_{\text{G}}\vec{H}$. Also, we define $c=\frac{1}{\sqrt{\varepsilon_{\text{G}}\mu_{\text{G}}}}$, $G=\frac{1}{4\pi\varepsilon_{\text{G}}}$, together with $\vec{E}=-\vec{\nabla}\varphi-\frac{\p}{\p t}\vec{A}$, $\vec{B}=\vec{\nabla}\times\vec{A}$, and $\bar{h}_{0 \mu}=-4A_\mu/c=-4(\varphi/c,\vec{A})/c$.
\end{document}
