\documentclass{article}
\usepackage{amsmath}
\def\d{\mathrm{d}}
\def\p{\partial}
\title{Constraining Speed of Gravitational Wave and {N}ewton's Constant in Binary Systems}
\author{An Jiachen}
\date{}
\begin{document}

\maketitle

\begin{abstract}
    Any constant varying in space and/or time will induce a violation of the universality of free fall. In this paper we provided a method to constrain speed of gravitational wave and Newton's constant in binary systems by comparing the results, which are obtained by using gravitational wave and electromagnetic wave observation data and methods in former researches, of measuring the luminosity distances of the binary systems. To strongly constrain speed of gravitational wave and Newton's constant by using method provided in this paper, results of using gravitational wave observation data to measure the luminosity distances in unreached high precision are needed.
\end{abstract}

\section{Introduction}

Any constant varying in space and/or time will induce a violation of the universality of free fall, and various experiments have been conducted on it \cite{Uzan2011}. Methods of constraining temporal variation of Newton's constant $G$ in binary systems with the assumption that speed of gravitational wave $c$ is constant have been provided \cite{Yunes2010}. In this paper we concentrate on spatial variations. We provided a method to constrain $c$ and $G$ in binary systems by comparing the results, which are obtained by using gravitational wave (GW) and electromagnetic wave (EMW) observation data and methods in former researches, of measuring the luminosity distances of the binary systems.

\section{Methods and Results}

If we only take the lowest order effects of the motion of mass sources into account and neglect stresses, then we can define two vector fields, $\vec{E}$ and $\vec{B}$, which are similar to the electric and magnetic field \cite{Wald1984}. We can define $A_\mu=-\frac{1}{4}c_0\bar{h}_{0 \mu}=(\varphi/c_0,\vec{A})$, then $A_\mu$, which are similar to the electromagnetic 4-potential, will satisfy
\begin{equation}
    \p^\nu\p_\nu A_\mu=-\frac{4\pi G_0}{c_0^2}J_\mu,\quad\p^\mu A_\mu=0,
\end{equation}
where $J_\mu=-c_0T_{0 \mu}/c_0^2=(c_0\rho,\vec{j})=\rho(c_0,\vec{v})$ is 4-momentum density. If we define $\vec{E}=-\vec{\nabla}\varphi-\frac{\p}{\p t}\vec{A}$, $\vec{B}=\vec{\nabla}\times\vec{A}$, and $\varepsilon_{\text{G}0}=\frac{1}{4\pi G_0}$, $\mu_{\text{G}0}=\frac{4\pi G_0}{c_0^2}$, then there will be
\begin{equation}\label{maxwell_0}
    \begin{cases}
        \vec{\nabla}\cdot(\varepsilon_{\text{G}0}\vec{E})=\rho,\\
        \vec{\nabla}\cdot\vec{B}=0,\\
        \vec{\nabla}\times\vec{E}=-\frac{\p}{\p t}\vec{B},\\
        \vec{\nabla}\times(\mu_{\text{G}0}^{-1}\vec{B})=\vec{j}+\frac{\p}{\p t}(\varepsilon_{\text{G}0}\vec{E}),
    \end{cases}
\end{equation}
and the acceleration of mass particle $\vec{a}$ will satisfy
\begin{equation}\label{lorentz}
    \vec{a}=-\vec{E}-4\vec{v}\times\vec{B}.
\end{equation}

Since \eqref{maxwell_0} are identical to Maxwell's equations, and \eqref{lorentz} is identical to the Lorentz force equation except for an overall minus sign and a factor of $4$ in the ``magnetic force'' term, we assume that the constants $\varepsilon_{\text{G}0}$ and $\mu_{\text{G}0}$ can simply vary as constants $\varepsilon_{0}$ and $\mu_{0}$ in electromagnetic theories, which means, \eqref{lorentz} is still valid and \eqref{maxwell_0} becomes
\begin{equation}\label{maxwell}
    \begin{cases}
        \vec{\nabla}\cdot\vec{D}=\rho,\\
        \vec{\nabla}\cdot\vec{B}=0,\\
        \vec{\nabla}\times\vec{E}=-\frac{\p}{\p t}\vec{B},\\
        \vec{\nabla}\times\vec{H}=\vec{j}+\frac{\p}{\p t}\vec{D},
    \end{cases}
\end{equation}
where $\vec{D}=\varepsilon_{\text{G}}\vec{E}$, $\vec{B}=\mu_{\text{G}}\vec{H}$. Also, we define $c=\frac{1}{\sqrt{\varepsilon_{\text{G}}\mu_{\text{G}}}}$, $G=\frac{1}{4\pi\varepsilon_{\text{G}}}$, together with $\vec{E}=-\vec{\nabla}\varphi-\frac{\p}{\p t}\vec{A}$, $\vec{B}=\vec{\nabla}\times\vec{A}$, and $\bar{h}_{0 \mu}=-4A_\mu/c=-4(\varphi/c,\vec{A})/c$.

We assume that the source of GW is located in a spherical coordinate origin and $c=c(r)$, $G=G(r)$, We suppose that $\bar{h}_{0 \mu}(\rho,\vec{j};c_0,G_0)$ is the solution of $\bar{h}_{0 \mu}$ when $\rho$ and $\vec{j}$ are assumed and $c\equiv c_0$, $G\equiv G_0$ everywhere, then if $c\equiv c_\text{s}$, $G\equiv G_\text{s}$, where $c_\text{s}$ and $G_\text{s}$ are two constants, within a region $r\leq R$, then $\bar{h}_{0 \mu}(\rho,\vec{j};c_\text{s},G_\text{s})$ can be a solution of $\bar{h}_{0 \mu}$ within the region $r\leq R$, and now we can solve the equations
\begin{equation}\label{maxwell_sf}
    \begin{cases}
        \vec{\nabla}\cdot\vec{D}=0,\\
        \vec{\nabla}\cdot\vec{B}=0,\\
        \vec{\nabla}\times\vec{E}=-\frac{\p}{\p t}\vec{B},\\
        \vec{\nabla}\times\vec{H}=+\frac{\p}{\p t}\vec{D},
    \end{cases}
\end{equation}
with the boundary conditions $\vec{E}|_{r=R}=\vec{E}(\rho,\vec{j};c_\text{s},G_\text{s})$, $\vec{B}|_{r=R}=\vec{B}(\rho,\vec{j};c_\text{s},G_\text{s})$, where $\vec{E}(\rho,\vec{j};c_\text{s},G_\text{s})$ and $\vec{B}(\rho,\vec{j};c_\text{s},G_\text{s})$ can be derived from $\bar{h}_{0 \mu}(\rho,\vec{j};c_\text{s},G_\text{s})$.

Again, since \eqref{maxwell_0} are identical to Maxwell's equations, we can use the Li\'enard--Wiechert potentials formula to calculate the $\vec{E}$ and $\vec{B}$ field produced by a moving particle, whose position is $\vec{r}'$, in vacuum, and the results are
\begin{equation}\label{lw}
    \vec{E}=\frac{m}{4 \pi \varepsilon_{\text{G}0}}\left[\frac{(\vec{n}'-\vec{\beta})}{(1-\vec{\beta} \cdot \vec{n}')^{3} \gamma^{2} {r'}^{2}}+\frac{\vec{n}' \times\{(\vec{n}'-\vec{\beta}) \times \dot{\vec{\beta}}\}}{(1-\vec{\beta} \cdot \vec{n}')^{3} c_0 r'}\right]_{\text{ret}},\vec{B}=\left[\frac{\vec{n}'\times \vec{E}}{c_0}\right]_{\text{ret}},
\end{equation}
where $r'=\lvert\vec{r}'\rvert$, $\vec{n}'=\vec{r}'/r'$, $\vec{\beta}=\dot{\vec{r}}'/c_0$, $\gamma=1/\sqrt{1-\lvert\vec{\beta}\rvert^2}$, and ``ret'' denotes that the fields are ``retarded fields''. Both $\vec{E}$ and $\vec{B}$ in \eqref{lw} can be divided into two parts: one, called ``inherent field part'', is proportional to $1/r'^2$ and the other one, called ``radiation part'', is proportional to $1/r'$. We only take the lowest order effects of the motion of mass sources into account, therefore we can assume that $\lvert\vec{\beta}\rvert\ll 1$. The ``inherent field part'' of $\vec{E}$ will be parallel to $\vec{r}'$ and the ``inherent field part'' of $\vec{B}$ will be nearly zero vector. The ``radiation part'' of $\vec{E}$ and $\vec{B}$ will be perpendicular to $\vec{r}'$. Since \eqref{maxwell_sf} are linear equations, we can discuss two parts of $\vec{E}$ and $\vec{B}$ separately for a binary GW source. If we assume that the distance between two components of binary is much less than $R$ mentioned above, then the ``inherent field part'' of $\vec{E}$ will have only $r$ component, and the ``radiation part'' of $\vec{E}$ and $\vec{B}$ will have no $r$ component.

For the ``inherent field part'', \eqref{maxwell_sf} will become $\frac{1}{r^2}\frac{\p}{\p r}(r^2 D_r)=0$, then $D_r\propto 1/r^2$, and $E_r\propto \varepsilon_{\text{G}}^{-1}(1/r^2)$. As mentioned below, for the ``radiation part'', $E\propto c^{-1}\varepsilon_{\text{G}}^{-1/2}(1/r)$, therefore the ``inherent field part'' will vanish in the region far away from source.

For the ``radiation part'', from the third and fourth equation of \eqref{maxwell_sf}, we can derive
\begin{equation}
    \begin{cases}
        [\vec{e}_r\cdot(\vec{\nabla}\times\vec{E})]\vec{e}_r-\frac{1}{r}\frac{\p}{\p r}(r{E}_\phi)\vec{e}_\theta+\frac{1}{r}\frac{\p}{\p r}(r{E}_\theta)\vec{e}_\phi=-\mu_{\text{G}}\frac{\p}{\p t}({H}_\theta\vec{e}_\theta+{H}_\phi\vec{e}_\phi),\\
        [\vec{e}_r\cdot(\vec{\nabla}\times\vec{H})]\vec{e}_r-\frac{1}{r}\frac{\p}{\p r}(r{H}_\phi)\vec{e}_\theta+\frac{1}{r}\frac{\p}{\p r}(r{H}_\theta)\vec{e}_\phi=+\varepsilon_{\text{G}}\frac{\p}{\p t}({E}_\theta\vec{e}_\theta+{E}_\phi\vec{e}_\phi),
    \end{cases}
\end{equation}
and then
\begin{equation}
    \begin{cases}
        \frac{1}{r}\frac{\p}{\p r}(r{E}_\theta)
        =-\mu_{\text{G}}\frac{\p}{\p t}{H}_\phi,\\
        \frac{1}{r}\frac{\p}{\p r}(r{E}_\phi)
        =+\mu_{\text{G}}\frac{\p}{\p t}{H}_\theta,\\
        \frac{1}{r}\frac{\p}{\p r}(r{H}_\theta)
        =+\varepsilon_{\text{G}}\frac{\p}{\p t}{E}_\phi,\\
        \frac{1}{r}\frac{\p}{\p r}(r{H}_\phi)
        =-\varepsilon_{\text{G}}\frac{\p}{\p t}{E}_\theta,\\
    \end{cases}
\end{equation}
\begin{equation}
    \begin{cases}
        \frac{\p}{\p r}(r{E}_\theta)
        =-\mu_{\text{G}}\frac{\p}{\p t}(r{H}_\phi),\\
        \frac{\p}{\p r}(r{E}_\phi)
        =+\mu_{\text{G}}\frac{\p}{\p t}(r{H}_\theta),\\
        \frac{\p}{\p r}(r{H}_\theta)
        =+\varepsilon_{\text{G}}\frac{\p}{\p t}(r{E}_\phi),\\
        \frac{\p}{\p r}(r{H}_\phi)
        =-\varepsilon_{\text{G}}\frac{\p}{\p t}(r{E}_\theta),\\
    \end{cases}
\end{equation}
\begin{equation}
    \begin{cases}
        \mu_{\text{G}}\frac{\p}{\p r}[\mu_{\text{G}}^{-1}\frac{\p}{\p r}(r{E}_\theta)]
        =-\mu_{\text{G}}\frac{\p}{\p t}\frac{\p}{\p r}(r{H}_\phi)
        =\varepsilon_{\text{G}}\mu_{\text{G}}\frac{\p}{\p t}\frac{\p}{\p t}(r{E}_\theta),\\
        \mu_{\text{G}}\frac{\p}{\p r}[\mu_{\text{G}}^{-1}\frac{\p}{\p r}(r{E}_\phi)]
        =+\mu_{\text{G}}\frac{\p}{\p t}\frac{\p}{\p r}(r{H}_\theta)
        =\varepsilon_{\text{G}}\mu_{\text{G}}\frac{\p}{\p t}\frac{\p}{\p t}(r{E}_\phi),\\
        \varepsilon_{\text{G}}\frac{\p}{\p r}[\varepsilon_{\text{G}}^{-1}\frac{\p}{\p r}(r{H}_\theta)]
        =+\mu_{\text{G}}\frac{\p}{\p t}\frac{\p}{\p r}(r{E}_\phi)
        =\varepsilon_{\text{G}}\mu_{\text{G}}\frac{\p}{\p t}\frac{\p}{\p t}(r{H}_\theta),\\
        \varepsilon_{\text{G}}\frac{\p}{\p r}[\varepsilon_{\text{G}}^{-1}\frac{\p}{\p r}(r{H}_\phi)]
        =-\mu_{\text{G}}\frac{\p}{\p t}\frac{\p}{\p r}(r{E}_\theta)
        =\varepsilon_{\text{G}}\mu_{\text{G}}\frac{\p}{\p t}\frac{\p}{\p t}(r{H}_\phi),\\
    \end{cases}
\end{equation}
therefore
\begin{equation}\label{radiation_part}
    \begin{cases}
        \frac{\p}{\p r}\frac{\p}{\p r}(r{E}_\theta)-\frac{\p}{\p r}(\ln\mu_{\text{G}})\frac{\p}{\p r}(r{E}_\theta)-\frac{\p}{\p (ct)}\frac{\p}{\p (ct)}(r{E}_\theta)=0,\\
        \frac{\p}{\p r}\frac{\p}{\p r}(r{E}_\phi)-\frac{\p}{\p r}(\ln\mu_{\text{G}})\frac{\p}{\p r}(r{E}_\phi)-\frac{\p}{\p (ct)}\frac{\p}{\p (ct)}(r{E}_\phi)=0,\\
        \frac{\p}{\p r}\frac{\p}{\p r}(r{H}_\theta)-\frac{\p}{\p r}(\ln\varepsilon_{\text{G}})\frac{\p}{\p r}(r{H}_\theta)-\frac{\p}{\p (ct)}\frac{\p}{\p (ct)}(r{H}_\theta)=0,\\
        \frac{\p}{\p r}\frac{\p}{\p r}(r{H}_\phi)-\frac{\p}{\p r}(\ln\varepsilon_{\text{G}})\frac{\p}{\p r}(r{H}_\phi)-\frac{\p}{\p (ct)}\frac{\p}{\p (ct)}(r{H}_\phi)=0.\\
    \end{cases}
\end{equation}
All equations in \eqref{radiation_part} have the form like
\begin{equation}\label{radiation}
    \frac{\p^2}{\p r^2}f(r,t)-p(r)\frac{\p}{\p r}f(r,t)-\frac{\p^2}{\p (ct)^2}f(r,t)=0.
\end{equation}
If we solve \eqref{radiation} by the method of separation of variables, we can suppose that $f(r,t)=f(r)e^{-ikct}$, and then
\begin{equation}\label{radiation_r}
    \frac{\d^2}{\d r^2}f(r)-p(r)\frac{\d}{\d r}f(r)+k^2f(r)=0.
\end{equation}
If $p$ is constant, then
\begin{equation}
    f(r)=e^{(p/2)r}[C_+e^{i\sqrt{k^2-(p/2)^2}r}+C_-e^{-i\sqrt{k^2-(p/2)^2}r}],
\end{equation}
where $C_+$ and $C_-$ are two constants, and
\begin{equation}
    f(r,t)=e^{(p/2)r}[C_+e^{i(+\sqrt{(\omega/c)^2-(p/2)^2}r-\omega t)}+C_-e^{i(-\sqrt{(\omega/c)^2-(p/2)^2}r-\omega t)}],
\end{equation}
where $\omega=kc$. If $\d p/\d r$ is small, we can divide the region $r> R$ into many spherical shells, and in each of them $p$ is nearly constant. To make $f(r,t)$ continuous anytime, we suppose that the ``angular frequency'' $\omega$ in each shell is the same, and then there will be a approximate solution
\begin{equation}\label{wave_solution}
    f(r,t)=e^{\int(p/2)\d r}[C_+e^{i(+\int\sqrt{(\omega/c)^2-(p/2)^2}\d r-\omega t)}+C_-e^{i(-\int\sqrt{(\omega/c)^2-(p/2)^2}\d r-\omega t)}],
\end{equation}
therefore
\begin{equation}
    f(r,t)=e^{\int O(p)\d r}[C_+e^{i(+\int[(\omega/c)^2+O(p^2)]\d r-\omega t)}+C_-e^{i(+\int[(\omega/c)^2+O(p^2)]\d r-\omega t)}].
\end{equation}
If $p$ is small, \eqref{wave_solution} can be approximated to
\begin{equation}
    f(r,t)=e^{\int(p/2)\d r}[C_+e^{i(+\int(\omega/c)\d r-\omega t)}+C_-e^{i(-\int(\omega/c)\d r-\omega t)}],
\end{equation}
then wave of $\vec{E}$ and $\vec{H}$ will approximately arrive Earth at the same time, and also $rE \propto e^{\int(\frac{\d}{\d r}(\ln\mu_{\text{G}})/2)\d r} \propto \mu_{\text{G}}^{1/2}$, $rH\propto e^{\int(\frac{\d}{\d r}(\ln\varepsilon_{\text{G}})/2)\d r} \propto \varepsilon_{\text{G}}^{1/2}$. Again, if $p$ is small, $\left\lvert i(\omega/c)\int f\d r\right\rvert \approx\left\lvert \frac{\p}{\p r}\int f\d r\right\rvert $ and $-i\omega\int f\d t=\frac{\p}{\p t}\int f\d t$. Since $\vec{H}=\mu_\text{G}^{-1}\vec{\nabla}\times\vec{A}$ and $H\propto \varepsilon_\text{G}^{1/2}/r$, $\mu_\text{G}^{-1}(\omega/c)A\propto\varepsilon_{\text{G}}^{1/2}/r\Rightarrow A\propto \mu_\text{G}^{1/2}/r$. Since $\vec{E}=-\vec{\nabla}\varphi-\frac{\p}{\p t}\vec{A}$, $E\propto \mu_\text{G}^{1/2}/r$ and $\frac{\p}{\p t}\vec{A}\propto \omega\mu_\text{G}^{1/2}/r\propto \mu_\text{G}^{1/2}/r$, $(\omega/c)\varphi\propto\mu_\text{G}^{1/2}/r\Rightarrow \varphi/c\propto\mu_\text{G}^{1/2}/r$. Therefore, $\bar{h}_{0 \mu}=-4(\varphi/c,\vec{A})/c\propto\mu_\text{G}^{1/2}/cr\propto G^{1/2}/c^2r$.

It is known that if $c\equiv c_0$, $G\equiv G_0$ everywhere, then the amplitude $h$ of GW produced by an inspiring binary satisfies
\begin{equation}
    h\propto\frac{\frac{G_0\mathcal{M}}{c_0^3}[\pi \frac{G_0\mathcal{M}}{c_0^3}F(t)]^{2/3}}{r/c_0}\cos[\int 2\pi F(t)\,\d t],
\end{equation}
where $\mathcal{M}$ is the chirp mass of binary and $F$ is the frequency of GW. Therefore in the region $r\leq R$ mentioned above,
\begin{equation}
    h\propto\frac{\frac{G_\text{s}\mathcal{M}}{c_\text{s}^3}[\pi \frac{G_\text{s}\mathcal{M}}{c_\text{s}^3}F(t)]^{2/3}}{r/c_\text{s}}\cos[\int 2\pi F(t)\,\d t],
\end{equation}
and then on Earth,
\begin{equation}
    h\propto\sqrt{\frac{c_\text{s}^4/G_\text{s}}{c_0^4/G_0}}\frac{\frac{G_\text{s}\mathcal{M}}{c_\text{s}^3}[\pi \frac{G_\text{s}\mathcal{M}}{c_\text{s}^3}F(t)]^{2/3}}{d/c_\text{s}}\cos[\int 2\pi F(t)\,\d t],
\end{equation}
where $d$ is the physical distance of binary. Considering the red shift, the amplitude of GW produced by an inspiring binary on Earth satisfies
\begin{equation}
    h\propto\sqrt{\frac{c_\text{s}^4/G_\text{s}}{c_0^4/G_0}}\frac{\frac{G_\text{s}\mathcal{M}}{c_\text{s}^3}[\pi \frac{G_\text{s}\mathcal{M}}{c_\text{s}^3}F(t)]^{2/3}}{d/c_\text{s}}\cos[\int 2\pi \frac{F(t)}{1+z}\,\d t],
\end{equation}
where $z$ is the red shift of binary. If we define $F_\text{obs}(t)=F(t)/(1+z)$ and $\mathcal{M}_\text{obs}=\mathcal{M}(1+z)$, then
\begin{equation}
    h\propto\sqrt{\frac{c_\text{s}^6/G_\text{s}}{c_0^6/G_0}}\frac{\frac{G_\text{s}\mathcal{M}_\text{obs}}{c_\text{s}^3}[\pi \frac{G_\text{s}\mathcal{M}_\text{obs}}{c_\text{s}^3}F_\text{obs}(t)]^{2/3}}{d_\text{L}/c_0}\cos[\int 2\pi F_\text{obs}(t)\,\d t],\label{new_h}
\end{equation}
where $d_\text{L}=d(1+z)$ is the luminosity distance of binary. However, in former researches, $h$ was assumed to satisfy
\begin{equation}
    h\propto\frac{\frac{G_0\mathcal{M}_\text{obs}}{c_0^3}[\pi \frac{G_0\mathcal{M}_\text{obs}}{c_0^3}F_\text{obs}(t)]^{2/3}}{d_\text{L}/c_0}\cos[\int 2\pi F_\text{obs}(t)\,\d t],\label{old_h}
\end{equation}
therefore, since ${G_\text{s}\mathcal{M}_\text{obs}}/{c_\text{s}^3}$ in \eqref{new_h} and ${G_0\mathcal{M}_\text{obs}}/{c_\text{s}^3}$ in \eqref{old_h} can be individually determined from $F_\text{obs}(t)$, $d_\text{L,G}$, the result of measuring $d_\text{L}$ by using GW observation data and methods in former researches, will satisfy
\begin{equation}
    \sqrt{\frac{G_\text{s}/c_\text{s}^6}{G_0/c_0^6}}=\frac{d_\text{L,G}}{d_\text{L}},
\end{equation}
where $d_\text{L}$ can also be measured precisely by using EMW observation data, if we assume that the speed of EMW is still $c_0$ everywhere.

Therefore, we can constrain $c_\text{s}$ and $G_\text{s}$ by comparing $d_\text{L}$ and $d_\text{L,G}$, and $\sqrt{[\frac{1}{2}\Delta\!\ln(G_\text{s}/G_0)]^2+[3\Delta\!\ln(c_\text{s}/c_0)]^2}=\Delta\!\ln(d_\text{L,G}/d_\text{L})$, where $\Delta$ denotes to the uncertainty. The result of measuring $d_\text{L}$ and $d_\text{L,G}$ for GW170817/GRB 170817A is $d_\text{L,G}/d_\text{L}=40_{-14}^{+8}\text{Mpc}/42.9_{-3.2}^{+3.2}\text{Mpc}$ \cite{Abbott2017}, and we estimate that this result could constrain $c_\text{s}$ and $G_\text{s}$ as $(G_\text{s}/G_0)^{1/2}(c_\text{s}/c_0)^3=0.9 \pm 0.3$.

\section{Conclusion}
In this paper, we provided a method to constrain speed of GW $c$ and Newton's constant $G$ in binary systems by comparing the results, which are obtained by using GW and EMW observation data and methods in former researches, of measuring the luminosity distances of the binary systems. This method can be practicable when \eqref{lorentz} and \eqref{maxwell} are vaild, together with:
\begin{enumerate}
    \item speeds of two components of binary are low;
    \item in a spherical region where binary is located at the center, $c$ and $G$ are nearly constants, and the radius of the region is much greater than the distance between two components of binary;
    \item both $c$ and $G$ are only related to $r$, the distance from binary;
    \item frequency of separate GW remains unchanged.
    \item $\frac{\d}{\d r}\ln c$, $\frac{\d}{\d r}\ln G$, $\frac{\d^2}{\d r^2}\ln c$, $\frac{\d^2}{\d r^2}\ln G$ are all small.
\end{enumerate}
The strength of constraining $c$ and $G$ by using method provided in this paper is depended on the uncertainties of the results of measuring the luminosity distances, especially that by using GW observation data. Therefore, to strongly constrain $c$ and $G$, results of using GW observation data to measure the luminosity distances in unreached high precision are needed.

\bibliographystyle{ieeetr}
\bibliography{bib}

\end{document}
